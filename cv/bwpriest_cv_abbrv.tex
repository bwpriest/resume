%%%%%%%%%%%%%%%%%%%%%%%%%%%%%%%%%%%%%%%%%%%%%%%%%%%%%%%%%%%%%%%%%%%%%%%%
%%%%%%%%%%%%%%%%%%%%%% Simple LaTeX CV Template %%%%%%%%%%%%%%%%%%%%%%%%
%%%%%%%%%%%%%%%%%%%%%%%%%%%%%%%%%%%%%%%%%%%%%%%%%%%%%%%%%%%%%%%%%%%%%%%%

%%%%%%%%%%%%%%%%%%%%%%%%%%%%%%%%%%%%%%%%%%%%%%%%%%%%%%%%%%%%%%%%%%%%%%%%
%% NOTE: If you find that it says                                     %%
%%                                                                    %%
%%                           1 of ??                                  %%
%%                                                                    %%
%% at the bottom of your first page, this means that the AUX file     %%
%% was not available when you ran LaTeX on this source. Simply RERUN  %%
%% LaTeX to get the ``??'' replaced with the number of the last page  %%
%% of the document. The AUX file will be generated on the first run   %%
%% of LaTeX and used on the second run to fill in all of the          %%
%% references.                                                        %%
%%%%%%%%%%%%%%%%%%%%%%%%%%%%%%%%%%%%%%%%%%%%%%%%%%%%%%%%%%%%%%%%%%%%%%%%

%%%%%%%%%%%%%%%%%%%%%%%%%%%% Document Setup %%%%%%%%%%%%%%%%%%%%%%%%%%%%

% Don't like 10pt? Try 11pt or 12pt
\documentclass[10pt]{article}
\RequirePackage[T1]{fontenc}

% LaTeX will typeset using Computer Modern Roman, which a lot of
% non-mathematicians and non-engineers won't like. Also, a few PDF
% viewers may not render CMR very well. Instead, Times New Roman can
% be used. That's what this package does.
\usepackage{times}

% The automated optical recognition software used to digitize resume
% information works best with fonts that do not have serifs. This
% command uses a sans serif font throughout. Uncomment both lines (or at
% least the second) to restore a Roman font (i.e., a font with serifs).
% (NOTE: This requires the times package above)
%\renewcommand{\familydefault}{\sfdefault}

% This is a helpful package that puts math inside length specifications
\usepackage{calc}

% This package helps LaTeX auto-hyphenate hyphenated words if you use
% special hyphens. For example, bio\-/mimicry will properly hyphenate
% ``mimicry'' if necessary.
\usepackage[shortcuts]{extdash}

% Layout: Puts the section titles on left side of page
\reversemarginpar

%
%         PAPER SIZE, PAGE NUMBER, AND DOCUMENT LAYOUT NOTES:
%
% The next \usepackage line changes the layout for CV style section
% headings as marginal notes. It also sets up the paper size as either
% letter or A4. By default, letter was used. If A4 paper is desired,
% comment out the letterpaper lines and uncomment the a4paper lines.
%
% As you can see, the margin widths and section title widths can be
% easily adjusted.
%
% ALSO: Notice that the includefoot option can be commented OUT in order
% to put the PAGE NUMBER *IN* the bottom margin. This will make the
% effective text area larger.
%
% IF YOU WISH TO REMOVE THE ``of LASTPAGE'' next to each page number,
% see the note about the +LP and -LP lines below. Comment out the +LP
% and uncomment the -LP.
%
% IF YOU WISH TO REMOVE PAGE NUMBERS, be sure that the includefoot line
% is uncommented and ALSO uncomment the \pagestyle{empty} a few lines
% below.
%

%% Use these lines for letter-sized paper
\usepackage[paper=letterpaper,
            %includefoot, % Uncomment to put page number above margin
            marginparwidth=1.2in,     % Length of section titles
            marginparsep=.05in,       % Space between titles and text
            margin=1in,               % 1 inch margins
            includemp]{geometry}

%% Use these lines for A4-sized paper
%\usepackage[paper=a4paper,
%            %includefoot, % Uncomment to put page number above margin
%            marginparwidth=30.5mm,    % Length of section titles
%            marginparsep=1.5mm,       % Space between titles and text
%            margin=25mm,              % 25mm margins
%            includemp]{geometry}

%% More layout: Get rid of indenting throughout entire document
\setlength{\parindent}{0in}

% Provides special list environments and macros to create new ones
\usepackage[shortlabels]{enumitem}

% Simpler bibsections for CV sections
% (thanks to natbib for inspiration)
%
% * For lists of references with hanging indents and no numbers:
%
%   \begin{bibsection}
%       \item ...
%   \end{bibsection}
%
% * For numbered lists of references (with hanging indents):
%
%   \begin{bibenum}
%       \item ...
%   \end{bibenum}
%
%   Note that bibenum numbers continuously throughout. To reset the
%   counter, use
%
%   \restartlist{bibenum}
%
%   at the place where you want the numbering to reset.


% Allows us to cite items inline
\usepackage[numbers]{natbib}
\usepackage{bibentry}

\nobibliography*

\newcommand{\ndd}{\textbf{In preparation}}
\newcommand{\myname}{\textbf{Benjamin W Priest}}

\makeatletter
\newlength{\bibbhang}
\setlength{\bibbhang}{1em}
\newlength{\bibbsep}
 {\@listi \global\bibbsep\itemsep \global\advance\bibbsep by\parsep}
\newlist{bibbsection}{itemize}{3}
\setlist[bibbsection]{label=,leftmargin=\bibbhang,%
        itemindent=-\bibbhang,
        itemsep=\bibbsep,parsep=\z@,partopsep=0pt,
        topsep=0pt}
\newlist{bibenum}{enumerate}{3}
\setlist[bibenum]{label=[\arabic*],resume,leftmargin={\bibbhang+\widthof{[999]}},%
        itemindent=-\bibbhang,
        itemsep=\bibbsep,parsep=\z@,partopsep=0pt,
        topsep=0pt}
\let\oldendbibenum\endbibenum
\def\endbibenum{\oldendbibenum\vspace{-.6\baselineskip}}
\let\oldendbibbsection\endbibbsection
\def\endbibbsection{\oldendbibbsection\vspace{-.6\baselineskip}}
\makeatother

%% Reference the last page in the page number
%
% NOTE: comment the +LP line and uncomment the -LP line to have page
%       numbers without the ``of ##'' last page reference)
%
% NOTE: uncomment the \pagestyle{empty} line to get rid of all page
%       numbers (make sure includefoot is commented out above)
%
\usepackage{fancyhdr,lastpage}
\pagestyle{fancy}
%\pagestyle{empty}      % Uncomment this to get rid of page numbers
\fancyhf{}\renewcommand{\headrulewidth}{0pt}
\fancyfootoffset{\marginparsep+\marginparwidth}
\newlength{\footpageshift}
\setlength{\footpageshift}
          {0.5\textwidth+0.5\marginparsep+0.5\marginparwidth-2in}
\lfoot{\hspace{\footpageshift}%
       \parbox{4in}{\, \hfill %
                    \arabic{page} of \protect\pageref*{LastPage} % +LP
%                    \arabic{page}                               % -LP
                    \hfill \,}}

% Finally, give us PDF bookmarks
\usepackage{color,hyperref}
\definecolor{darkblue}{rgb}{0.0,0.0,0.3}
\hypersetup{colorlinks,breaklinks,
            linkcolor=darkblue,urlcolor=darkblue,
            anchorcolor=darkblue,citecolor=darkblue}

%%%%%%%%%%%%%%%%%%%%%%%% End Document Setup %%%%%%%%%%%%%%%%%%%%%%%%%%%%


%%%%%%%%%%%%%%%%%%%%%%%%%%% Helper Commands %%%%%%%%%%%%%%%%%%%%%%%%%%%%

%%% HEADING AT TOP OF CURRICULUM VITAE

% The title (name) with a horizontal rule under it
% (optional argument typesets an object right-justified across from name
%  as well)
%
% Usage: \makeheading{name}
%        OR
%        \makeheading[right_object]{name}
%
% Place at top of document. It should be the first thing.
% If ``right_object'' is provided in the square-braced optional
% argument, it will be right justified on the same line as ``name'' at
% the top of the CV. For example:
%
%       \makeheading[\emph{Curriculum vitae}]{Your Name}
%
% will put an emphasized ``Curriculum vitae'' at the top of the document
% as a title. Likewise, a picture could be included:
%
%   \makeheading[{\includegraphics[height=1.5in]{my_picture}}]{Your Name}
%
% the picture will be flush right across from the name. For this example
% to work, make sure the extra set of curly braces is included. Also
% makes ure that \usepackage{graphicx} is somewhere in the preamble.
\newcommand{\makeheading}[2][]%
        {\hspace*{-\marginparsep minus \marginparwidth}%
         \begin{minipage}[t]{\textwidth+\marginparwidth+\marginparsep}%
             {\large \bfseries #2 \hfill #1}\\[-0.15\baselineskip]%
                 \rule{\columnwidth}{1pt}%
         \end{minipage}}

%%% SECTION HEADINGS

% The section headings. Flush left in small caps down pseudo-margin.
%
% Usage: \section{section name}
\renewcommand{\section}[1]{\pagebreak[3]%
    \vspace{1.3\baselineskip}%
    \phantomsection\addcontentsline{toc}{section}{#1}%
    \noindent\llap{\scshape\smash{\parbox[t]{\marginparwidth}{\hyphenpenalty=10000\raggedright #1}}}%
    \vspace{-\baselineskip}\par}

%%% LISTS

% This macro alters a list by removing some of the space that follows the list
% (is used by lists below)
\newcommand*\fixendlist[1]{%
    \expandafter\let\csname preFixEndListend#1\expandafter\endcsname\csname end#1\endcsname
    \expandafter\def\csname end#1\endcsname{\csname preFixEndListend#1\endcsname\vspace{-0.6\baselineskip}}}

% These macros help ensure that items in outer-type lists do not get
% separated from the next line by a page break
% (they are used by lists below)
\let\originalItem\item
\newcommand*\fixouterlist[1]{%
    \expandafter\let\csname preFixOuterList#1\expandafter\endcsname\csname #1\endcsname
    \expandafter\def\csname #1\endcsname{\let\oldItem\item\def\item{\pagebreak[2]\oldItem}\csname preFixOuterList#1\endcsname}
    \expandafter\let\csname preFixOuterListend#1\expandafter\endcsname\csname end#1\endcsname
    \expandafter\def\csname end#1\endcsname{\let\item\oldItem\csname preFixOuterListend#1\endcsname}}
\newcommand*\fixinnerlist[1]{%
    \expandafter\let\csname preFixInnerList#1\expandafter\endcsname\csname #1\endcsname
    \expandafter\def\csname #1\endcsname{\let\oldItem\item\let\item\originalItem\csname preFixInnerList#1\endcsname}
    \expandafter\let\csname preFixInnerListend#1\expandafter\endcsname\csname end#1\endcsname
    \expandafter\def\csname end#1\endcsname{\csname preFixInnerListend#1\endcsname\let\item\oldItem}}

% An itemize-style list with lots of space between items
%
% Usage:
%   \begin{outerlist}
%       \item ...    % (or \item[] for no bullet)
%   \end{outerlist}
\newlist{outerlist}{itemize}{3}
    \setlist[outerlist]{label=\enskip\textbullet,leftmargin=*}
    \fixendlist{outerlist}
    \fixouterlist{outerlist}

% An environment IDENTICAL to outerlist that has better pre-list spacing
% when used as the first thing in a \section
%
% Usage:
%   \begin{lonelist}
%       \item ...    % (or \item[] for no bullet)
%   \end{lonelist}
\newlist{lonelist}{itemize}{3}
    \setlist[lonelist]{label=\enskip\textbullet,leftmargin=*,partopsep=0pt,topsep=0pt}
    \fixendlist{lonelist}
    \fixouterlist{lonelist}

% An itemize-style list with little space between items
%
% Usage:
%   \begin{innerlist}
%       \item ...    % (or \item[] for no bullet)
%   \end{innerlist}
\newlist{innerlist}{itemize}{3}
    \setlist[innerlist]{label=\enskip\textbullet,leftmargin=*,parsep=0pt,itemsep=0pt,topsep=0pt,partopsep=0pt}
    \fixinnerlist{innerlist}

% An environment IDENTICAL to innerlist that has better pre-list spacing
% when used as the first thing in a \section
%
% Usage:
%   \begin{loneinnerlist}
%       \item ...    % (or \item[] for no bullet)
%   \end{loneinnerlist}
\newlist{loneinnerlist}{itemize}{3}
    \setlist[loneinnerlist]{label=\enskip\textbullet,leftmargin=*,parsep=0pt,itemsep=0pt,topsep=0pt,partopsep=0pt}
    \fixendlist{loneinnerlist}
    \fixinnerlist{loneinnerlist}

%%% EXTRA SPACE

% To add some paragraph space between lines.
% This also tells LaTeX to preferably break a page on one of these gaps
% if there is a needed pagebreak nearby.
\newcommand{\blankline}{\quad\pagebreak[3]}
\newcommand{\halfblankline}{\quad\vspace{-0.5\baselineskip}\pagebreak[3]}

%%% FORMATTING MACROS

% Provides a linked \doi{#1} that links doi:#1 to http://dx.doi.org/#1
\usepackage{doi}
% To change the text before the DOI, adjust this command
%\renewcommand\doitext{doi:}

% Provides a linked \url{#1} that doesn't require escape characters
\usepackage{url}

% You can adjust the style \url{} uses here:
% (options are: same, rm, sf, tt; defaults to tt)
\urlstyle{same}

% For \email{ADDRESS}, links ADDRESS to the url mailto:ADDRESS
% (uncomment to typeset the e\-/mail address in typewriter font;
%  otherwise, will be typeset in the \urlstyle above)
%\DeclareUrlCommand\emaillink{\urlstyle{tt}}
\providecommand*\emaillink[1]{\nolinkurl{#1}}
\providecommand*\email[1]{\href{mailto:#1}{\emaillink{#1}}}

\providecommand\BibTeX{{B\kern-.05em{\sc i\kern-.025em b}\kern-.08em \TeX}}
\providecommand\Matlab{\textsc{Matlab}}

% Custom hyphenation rules for words that LaTeX has trouble with
\hyphenation{bio-mim-ic-ry bio-in-spi-ra-tion re-us-a-ble pro-vid-er Media-Wiki}

%%%%%%%%%%%%%%%%%%%%%%%% End Helper Commands %%%%%%%%%%%%%%%%%%%%%%%%%%%

%%%%%%%%%%%%%%%%%%%%%%%%% Begin CV Document %%%%%%%%%%%%%%%%%%%%%%%%%%%%

\begin{document}
\makeheading{~Benjamin~(Min) Wesley~Priest (they/them)}

%%---------------------------------------------------------------------------------------------------------
\section{Contact Information}
%%---------------------------------------------------------------------------------------------------------

% NOTE: Mind where the & separators and \\ breaks are in the following
%       table. Table is one row made up of three parboxes. The left
%       parbox has address info, the middle parbox has a vertical bar,
%       and the right parbox has phone and electronic contact
%       information.
%
% MACROS: \rcollength is the width of the right column of the table
%             (adjust it to your liking; default is 1.85in).
%         \spacewidth is width of area between left and right boxes.
%
\newlength{\rcollength}\setlength{\rcollength}{1.85in}%
\newlength{\spacewidth}\setlength{\spacewidth}{20pt}
%
\begin{tabular}[t]{@{}p{\textwidth-\rcollength-3.5\spacewidth}@{}p{\spacewidth}@{}p{\rcollength}}%
  \parbox{\textwidth-\rcollength-\spacewidth}{%
  Computing Scientist                                                  \\
  \href{https://computing.llnl.gov/casc/}{Center for Applied Scientific
  Computing}                                                           \\
  \href{https://www.llnl.gov/}{Lawrence Livermore National Laboratory} \\
  }

   &

  \parbox{1.5\rcollength}{\textit{github:}
  \href{https://github.com/bwpriest}{https://github.com/bwpriest}      \\
  \textit{E-mail:} \email{priest2@llnl.gov}                            \\
  }

\end{tabular}
\vspace{-1em}

\section{Research Interests}
\textbf{Efficient analysis of large, dynamic datasets:}
sketching,
streaming algorithms,
machine learning,
high performance computing,
graph algorithms,
numerical linear algebra,
compressed sensing,
graph theory,
optimization,
network analysis,
%complexity,
%automata,
and theory of deep learning.


\vspace{-0.5em}

\section{Education}
\href{https://engineering.dartmouth.edu/}{\textbf{Thayer School of Engineering}}
at 
\href{https://dartmouth.edu}{\textbf{Dartmouth College}},
Hanover, VT, USA
\vspace{-0.5em}
\begin{outerlist}

\vspace{-0.25em}

\item[] Ph.D.,
             Engineering
             %
\hfill \textbf{09/2015 -- 02/2019}

\vspace{-0.5em}
        \begin{innerlist}
%        \item Thesis Topic: \emph{Design and Analysis of Optimal Task-Processing Agents}
%        \item Thesis Proposal: \emph{Cooperative Task Processing}
%        \item Candidacy: \emph{Research Problems in Distributed Control for Energy Systems}
        \item[-] Advisor:
              \href{http://www.dartmouth.edu/~gvc/}
                   {Professor George~Cybenko}
        \item[-] Thesis: Sublinear Approximations of Vertex Centrality in Evolving Graphs
%        \item[-] Demonstrated novel, sublinear-space sketching algorithms to efficiently estimate local triangle counts and vertex centrality on large, distributed graphs
%        \item[-] Implemented novel HPC algorithms on cutting-edge architectures 
        \end{innerlist}
\end{outerlist}

%\blankline
\vspace{0.5em}


\href{http://www.osu.edu/}{\textbf{The Ohio State University}},
Columbus, OH, USA
\vspace{-0.5em}
\begin{outerlist}

\vspace{-0.25em}
\item[] B.S.,
        \href{http://www.math.osu.edu/}
             {Mathematics}
	\hfill \textbf{09/2007 -- 06/2011}
%        \begin{innerlist}
%        \item \emph{Cum Laude}, With Honors in Mathematics
%        \item Minor in \href{http://www.cse.ohio-state.edu/}
%                            {Philosophy}
%              (logics and formal languages)
%        \end{innerlist}
\vspace{-0.75em}
\item[] B.S.,
        \href{http://www.cse.osu.edu/}
             {Computer and Information Science}
             \hfill \textbf{09/2007 -- 06/2011}

%        \begin{innerlist}
%        \item \emph{Cum Laude}
%        \item Theory specialization (emphasis on computability and complexity)
%        \end{innerlist}

\end{outerlist}

\vspace{-0.75em}

\section{Research Experience}
\href{https://www.llnl.gov/}{\textbf{Lawrence Livermore National Laboratory}},
Livermore, CA, USA

\href{https://computation.llnl.gov/casc/}{\textbf{Center for Applied Scientific Computing}}.
Supervisors:
\href{https://people.llnl.gov/sanders29}{Geoff Sanders},
\href{https://pls.llnl.gov/people/staff-bios/physics/schneider-m}
{Michael Schneider}
and
\href{https://people.llnl.gov/pearce7}{Roger Pearce}

\vspace{-0.5em}

\begin{outerlist}

  \vspace{-0.5em}

  \item[] \textbf{\textit{Computing Scientist}}\hfill \textbf{02/2021 -- present}

  \vspace{-0.5em}

  PI and Co-I of multiple research projects investigating scalable graph
  analytics, machine learning, and statistical modeling on High-Performance
  Computing (HPC) systems.
  Supervised 1 postdoc and 8 graduate students.
  Selected research contributions include novel algorithms and software for
  scalable Gaussian process (GP) estimation~\cite{muyskens2021muygps}, cosmology,
  climate, and space domain modeling~\cite{muyskens2022star}, distributed
  subspace embedding and sketches~\cite{priest2020scaling}, and distributed K
  nearest neighbors.

  \vspace{-0.5em}

  \item[] \textbf{\textit{Postdoctoral Researcher}}%
  \hfill \textbf{04/2019 -- 02/2021}

  \vspace{-0.5em}

  Developed novel sketching algorithms to cluster \cite{priest2020scaling} and
  perform local query approximation \cite{priest2020degreesketch} massive graphs
  on HPC.
  Solved reinforcement learning \cite{goumiri2020reinforcement}, image
  classification \cite{goumiri2020star}, and quantum machine learning
  \cite{otten2020quantum} problems using GPs and neural kernels.

  \vspace{-0.5em}

  \item[] \textbf{\textit{Computation Student Intern}}%
  \hfill \textbf{05/2018 -- 01/2019}

  \vspace{-0.5em}

  Designed novel HPC communication library to accelerate non-traditional
  communications \cite{priest2019you}.
  Used cardinality sketches to estimate local triangle counts in distributed
  graphs \cite{priest2018estimating}.

\end{outerlist}


\halfblankline

\href{https://dartmouth.edu}{\textbf{Dartmouth College}},
Hanover, NH, USA

\href{https://engineering.dartmouth.edu/}{\textbf{Thayer School of Engineering}}.
Advisor:               
\href{http://www.dartmouth.edu/~gvc/}
{Professor George~Cybenko}

\vspace{-0.5em}

\begin{outerlist}

\vspace{-0.5em}

\item[] \textbf{\textit{Research and Teaching Assistant}}%
\hfill \textbf{09/2015 -- 02/2019}
%\vspace{-0.25em}

\vspace{-0.5em}

Invented streaming approximation algorithms for several centrality indices on massive graphs using sketches.
Designed game and graph-theoretic models  for advanced persistent threats in cyber defense.
Taught courses in machine learning and lead a team of TAs.

\end{outerlist}


\halfblankline

\href{http://www.ll.mit.edu/}{\textbf{MIT Lincoln Laboratory}},
Lexington, MA, USA

\href{https://www.ll.mit.edu/mission/cybersec/CADS/CADS.html}
{Cyber Analytics and Decision Systems}.
Supervisor:
\href{https://www.ll.mit.edu/mission/cybersec/cybersec-bios/carter-bio.html}
{Dr. Kevin M.~Carter}

\vspace{-0.5em}

\begin{outerlist}

\vspace{-0.5em}

\item[] {\normalsize \textbf{\textit{Assistant Research Scientist}}}%
\hfill \textbf{08/2011 -- 07/2015}

\vspace{-0.5em}

Modeled computer networks using novel machine learning algorithms.
Developed multi-agent systems for high-fidelity network simulations and cyber defense evaluation.

\end{outerlist}

\vspace{-0.5em}

\section{Awards}
\begin{innerlist}

  \item[-] \href{https://graphchallenge.mit.edu/champions}{HPEC Graph Challenge}
  Champion, 2020
  \item[-] \href{https://graphchallenge.mit.edu/champions}{HPEC Graph Challenge}
  Champion, 2019
  \item[-] \href{http://www.secrypt.icete.org/PreviousAwards.aspx}{SECRYPT} Best
  Paper Award, 2018

\end{innerlist}

\section{Most Closely Related Publications}
\begin{innerlist}
  \item[\cite{priest2020scaling}] \bibentry{priest2020scaling}
  \item[\cite{priest2020degreesketch}] \bibentry{priest2020degreesketch}
  \item[\cite{priest2019you}] \bibentry{priest2019you}
  \item[\cite{steil2021tripoll}] \bibentry{steil2021tripoll}
  \item[\cite{pearce2019one}] \bibentry{pearce2019one}
\end{innerlist}

\section{Other Publications}
\begin{innerlist}
  \item[\cite{muyskens2021muygps}] \bibentry{muyskens2021muygps}
  \item[\cite{muyskens2022star}] \bibentry{muyskens2022star}
  \item[\cite{priest2018estimating}] \bibentry{priest2018estimating}
  \item[\cite{dunton2022fast}] \bibentry{dunton2022fast}
  \item[\cite{steil2019distributed}] \bibentry{steil2019distributed}
\end{innerlist}

\bibliographystyle{unsrtnat}
\nobibliography{../bibs/pubs.bib}

\end{document}

%%%%%%%%%%%%%%%%%%%%%%%%%% End CV Document %%%%%%%%%%%%%%%%%%%%%%%%%%%%%

%----------------------------------------------------------------------%
% The following is copyright and licensing information for
% redistribution of this LaTeX source code; it also includes a liability
% statement. If this source code is not being redistributed to others,
% it may be omitted. It has no effect on the function of the above code.
%----------------------------------------------------------------------%
% Copyright (c) 2007, 2008, 2009, 2010, 2011 by Theodore P. Pavlic
%
% Unless otherwise expressly stated, this work is licensed under the
% Creative Commons Attribution-Noncommercial 3.0 United States License. To
% view a copy of this license, visit
% http://creativecommons.org/licenses/by-nc/3.0/us/ or send a letter to
% Creative Commons, 171 Second Street, Suite 300, San Francisco,
% California, 94105, USA.
%
% THE SOFTWARE IS PROVIDED "AS IS", WITHOUT WARRANTY OF ANY KIND, EXPRESS
% OR IMPLIED, INCLUDING BUT NOT LIMITED TO THE WARRANTIES OF
% MERCHANTABILITY, FITNESS FOR A PARTICULAR PURPOSE AND NONINFRINGEMENT.
% IN NO EVENT SHALL THE AUTHORS OR COPYRIGHT HOLDERS BE LIABLE FOR ANY
% CLAIM, DAMAGES OR OTHER LIABILITY, WHETHER IN AN ACTION OF CONTRACT,
% TORT OR OTHERWISE, ARISING FROM, OUT OF OR IN CONNECTION WITH THE
% SOFTWARE OR THE USE OR OTHER DEALINGS IN THE SOFTWARE.
%----------------------------------------------------------------------%
