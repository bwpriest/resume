%%%%%%%%%%%%%%%%%%%%%%%%%%%%%%%%%%%%%%%%%%%%%%%%%%%%%%%%%%%%%%%%%%%%%%%%
%%%%%%%%%%%%%%%%%%%%%% Simple LaTeX CV Template %%%%%%%%%%%%%%%%%%%%%%%%
%%%%%%%%%%%%%%%%%%%%%%%%%%%%%%%%%%%%%%%%%%%%%%%%%%%%%%%%%%%%%%%%%%%%%%%%

%%%%%%%%%%%%%%%%%%%%%%%%%%%%%%%%%%%%%%%%%%%%%%%%%%%%%%%%%%%%%%%%%%%%%%%%
%% NOTE: If you find that it says                                     %%
%%                                                                    %%
%%                           1 of ??                                  %%
%%                                                                    %%
%% at the bottom of your first page, this means that the AUX file     %%
%% was not available when you ran LaTeX on this source. Simply RERUN  %%
%% LaTeX to get the ``??'' replaced with the number of the last page  %%
%% of the document. The AUX file will be generated on the first run   %%
%% of LaTeX and used on the second run to fill in all of the          %%
%% references.                                                        %%
%%%%%%%%%%%%%%%%%%%%%%%%%%%%%%%%%%%%%%%%%%%%%%%%%%%%%%%%%%%%%%%%%%%%%%%%

%%%%%%%%%%%%%%%%%%%%%%%%%%%% Document Setup %%%%%%%%%%%%%%%%%%%%%%%%%%%%

% Don't like 10pt? Try 11pt or 12pt
\documentclass[10pt]{article}
\RequirePackage[T1]{fontenc}

% LaTeX will typeset using Computer Modern Roman, which a lot of
% non-mathematicians and non-engineers won't like. Also, a few PDF
% viewers may not render CMR very well. Instead, Times New Roman can
% be used. That's what this package does.
\usepackage{times}

% The automated optical recognition software used to digitize resume
% information works best with fonts that do not have serifs. This
% command uses a sans serif font throughout. Uncomment both lines (or at
% least the second) to restore a Roman font (i.e., a font with serifs).
% (NOTE: This requires the times package above)
%\renewcommand{\familydefault}{\sfdefault}

% This is a helpful package that puts math inside length specifications
\usepackage{calc}

% This package helps LaTeX auto-hyphenate hyphenated words if you use
% special hyphens. For example, bio\-/mimicry will properly hyphenate
% ``mimicry'' if necessary.
\usepackage[shortcuts]{extdash}

% Layout: Puts the section titles on left side of page
\reversemarginpar

%
%         PAPER SIZE, PAGE NUMBER, AND DOCUMENT LAYOUT NOTES:
%
% The next \usepackage line changes the layout for CV style section
% headings as marginal notes. It also sets up the paper size as either
% letter or A4. By default, letter was used. If A4 paper is desired,
% comment out the letterpaper lines and uncomment the a4paper lines.
%
% As you can see, the margin widths and section title widths can be
% easily adjusted.
%
% ALSO: Notice that the includefoot option can be commented OUT in order
% to put the PAGE NUMBER *IN* the bottom margin. This will make the
% effective text area larger.
%
% IF YOU WISH TO REMOVE THE ``of LASTPAGE'' next to each page number,
% see the note about the +LP and -LP lines below. Comment out the +LP
% and uncomment the -LP.
%
% IF YOU WISH TO REMOVE PAGE NUMBERS, be sure that the includefoot line
% is uncommented and ALSO uncomment the \pagestyle{empty} a few lines
% below.
%

%% Use these lines for letter-sized paper
\usepackage[paper=letterpaper,
            %includefoot, % Uncomment to put page number above margin
            marginparwidth=1.2in,     % Length of section titles
            marginparsep=.05in,       % Space between titles and text
            margin=1in,               % 1 inch margins
            includemp]{geometry}

%% Use these lines for A4-sized paper
%\usepackage[paper=a4paper,
%            %includefoot, % Uncomment to put page number above margin
%            marginparwidth=30.5mm,    % Length of section titles
%            marginparsep=1.5mm,       % Space between titles and text
%            margin=25mm,              % 25mm margins
%            includemp]{geometry}

%% More layout: Get rid of indenting throughout entire document
\setlength{\parindent}{0in}

% Provides special list environments and macros to create new ones
\usepackage[shortlabels]{enumitem}

% Simpler bibsections for CV sections
% (thanks to natbib for inspiration)
%
% * For lists of references with hanging indents and no numbers:
%
%   \begin{bibsection}
%       \item ...
%   \end{bibsection}
%
% * For numbered lists of references (with hanging indents):
%
%   \begin{bibenum}
%       \item ...
%   \end{bibenum}
%
%   Note that bibenum numbers continuously throughout. To reset the
%   counter, use
%
%   \restartlist{bibenum}
%
%   at the place where you want the numbering to reset.


% Allows us to cite items inline
\usepackage{natbib}
\usepackage{bibentry}

\nobibliography*

\newcommand{\ndd}{\textbf{In preparation}}
\newcommand{\myname}{\textbf{Benjamin W Priest}}

\makeatletter
\newlength{\bibbhang}
\setlength{\bibbhang}{1em}
\newlength{\bibbsep}
 {\@listi \global\bibbsep\itemsep \global\advance\bibbsep by\parsep}
\newlist{bibbsection}{itemize}{3}
\setlist[bibbsection]{label=,leftmargin=\bibbhang,%
        itemindent=-\bibbhang,
        itemsep=\bibbsep,parsep=\z@,partopsep=0pt,
        topsep=0pt}
\newlist{bibenum}{enumerate}{3}
\setlist[bibenum]{label=[\arabic*],resume,leftmargin={\bibbhang+\widthof{[999]}},%
        itemindent=-\bibbhang,
        itemsep=\bibbsep,parsep=\z@,partopsep=0pt,
        topsep=0pt}
\let\oldendbibenum\endbibenum
\def\endbibenum{\oldendbibenum\vspace{-.6\baselineskip}}
\let\oldendbibbsection\endbibbsection
\def\endbibbsection{\oldendbibbsection\vspace{-.6\baselineskip}}
\makeatother

%% Reference the last page in the page number
%
% NOTE: comment the +LP line and uncomment the -LP line to have page
%       numbers without the ``of ##'' last page reference)
%
% NOTE: uncomment the \pagestyle{empty} line to get rid of all page
%       numbers (make sure includefoot is commented out above)
%
\usepackage{fancyhdr,lastpage}
\pagestyle{fancy}
%\pagestyle{empty}      % Uncomment this to get rid of page numbers
\fancyhf{}\renewcommand{\headrulewidth}{0pt}
\fancyfootoffset{\marginparsep+\marginparwidth}
\newlength{\footpageshift}
\setlength{\footpageshift}
          {0.5\textwidth+0.5\marginparsep+0.5\marginparwidth-2in}
\lfoot{\hspace{\footpageshift}%
       \parbox{4in}{\, \hfill %
                    \arabic{page} of \protect\pageref*{LastPage} % +LP
%                    \arabic{page}                               % -LP
                    \hfill \,}}

% Finally, give us PDF bookmarks
\usepackage{color,hyperref}
\definecolor{darkblue}{rgb}{0.0,0.0,0.3}
\hypersetup{colorlinks,breaklinks,
            linkcolor=darkblue,urlcolor=darkblue,
            anchorcolor=darkblue,citecolor=darkblue}

%%%%%%%%%%%%%%%%%%%%%%%% End Document Setup %%%%%%%%%%%%%%%%%%%%%%%%%%%%


%%%%%%%%%%%%%%%%%%%%%%%%%%% Helper Commands %%%%%%%%%%%%%%%%%%%%%%%%%%%%

%%% HEADING AT TOP OF CURRICULUM VITAE

% The title (name) with a horizontal rule under it
% (optional argument typesets an object right-justified across from name
%  as well)
%
% Usage: \makeheading{name}
%        OR
%        \makeheading[right_object]{name}
%
% Place at top of document. It should be the first thing.
% If ``right_object'' is provided in the square-braced optional
% argument, it will be right justified on the same line as ``name'' at
% the top of the CV. For example:
%
%       \makeheading[\emph{Curriculum vitae}]{Your Name}
%
% will put an emphasized ``Curriculum vitae'' at the top of the document
% as a title. Likewise, a picture could be included:
%
%   \makeheading[{\includegraphics[height=1.5in]{my_picture}}]{Your Name}
%
% the picture will be flush right across from the name. For this example
% to work, make sure the extra set of curly braces is included. Also
% makes ure that \usepackage{graphicx} is somewhere in the preamble.
\newcommand{\makeheading}[2][]%
        {\hspace*{-\marginparsep minus \marginparwidth}%
         \begin{minipage}[t]{\textwidth+\marginparwidth+\marginparsep}%
             {\large \bfseries #2 \hfill #1}\\[-0.15\baselineskip]%
                 \rule{\columnwidth}{1pt}%
         \end{minipage}}

%%% SECTION HEADINGS

% The section headings. Flush left in small caps down pseudo-margin.
%
% Usage: \section{section name}
\renewcommand{\section}[1]{\pagebreak[3]%
    \vspace{1.3\baselineskip}%
    \phantomsection\addcontentsline{toc}{section}{#1}%
    \noindent\llap{\scshape\smash{\parbox[t]{\marginparwidth}{\hyphenpenalty=10000\raggedright #1}}}%
    \vspace{-\baselineskip}\par}

%%% LISTS

% This macro alters a list by removing some of the space that follows the list
% (is used by lists below)
\newcommand*\fixendlist[1]{%
    \expandafter\let\csname preFixEndListend#1\expandafter\endcsname\csname end#1\endcsname
    \expandafter\def\csname end#1\endcsname{\csname preFixEndListend#1\endcsname\vspace{-0.6\baselineskip}}}

% These macros help ensure that items in outer-type lists do not get
% separated from the next line by a page break
% (they are used by lists below)
\let\originalItem\item
\newcommand*\fixouterlist[1]{%
    \expandafter\let\csname preFixOuterList#1\expandafter\endcsname\csname #1\endcsname
    \expandafter\def\csname #1\endcsname{\let\oldItem\item\def\item{\pagebreak[2]\oldItem}\csname preFixOuterList#1\endcsname}
    \expandafter\let\csname preFixOuterListend#1\expandafter\endcsname\csname end#1\endcsname
    \expandafter\def\csname end#1\endcsname{\let\item\oldItem\csname preFixOuterListend#1\endcsname}}
\newcommand*\fixinnerlist[1]{%
    \expandafter\let\csname preFixInnerList#1\expandafter\endcsname\csname #1\endcsname
    \expandafter\def\csname #1\endcsname{\let\oldItem\item\let\item\originalItem\csname preFixInnerList#1\endcsname}
    \expandafter\let\csname preFixInnerListend#1\expandafter\endcsname\csname end#1\endcsname
    \expandafter\def\csname end#1\endcsname{\csname preFixInnerListend#1\endcsname\let\item\oldItem}}

% An itemize-style list with lots of space between items
%
% Usage:
%   \begin{outerlist}
%       \item ...    % (or \item[] for no bullet)
%   \end{outerlist}
\newlist{outerlist}{itemize}{3}
    \setlist[outerlist]{label=\enskip\textbullet,leftmargin=*}
    \fixendlist{outerlist}
    \fixouterlist{outerlist}

% An environment IDENTICAL to outerlist that has better pre-list spacing
% when used as the first thing in a \section
%
% Usage:
%   \begin{lonelist}
%       \item ...    % (or \item[] for no bullet)
%   \end{lonelist}
\newlist{lonelist}{itemize}{3}
    \setlist[lonelist]{label=\enskip\textbullet,leftmargin=*,partopsep=0pt,topsep=0pt}
    \fixendlist{lonelist}
    \fixouterlist{lonelist}

% An itemize-style list with little space between items
%
% Usage:
%   \begin{innerlist}
%       \item ...    % (or \item[] for no bullet)
%   \end{innerlist}
\newlist{innerlist}{itemize}{3}
    \setlist[innerlist]{label=\enskip\textbullet,leftmargin=*,parsep=0pt,itemsep=0pt,topsep=0pt,partopsep=0pt}
    \fixinnerlist{innerlist}

% An environment IDENTICAL to innerlist that has better pre-list spacing
% when used as the first thing in a \section
%
% Usage:
%   \begin{loneinnerlist}
%       \item ...    % (or \item[] for no bullet)
%   \end{loneinnerlist}
\newlist{loneinnerlist}{itemize}{3}
    \setlist[loneinnerlist]{label=\enskip\textbullet,leftmargin=*,parsep=0pt,itemsep=0pt,topsep=0pt,partopsep=0pt}
    \fixendlist{loneinnerlist}
    \fixinnerlist{loneinnerlist}

%%% EXTRA SPACE

% To add some paragraph space between lines.
% This also tells LaTeX to preferably break a page on one of these gaps
% if there is a needed pagebreak nearby.
\newcommand{\blankline}{\quad\pagebreak[3]}
\newcommand{\halfblankline}{\quad\vspace{-0.5\baselineskip}\pagebreak[3]}

%%% FORMATTING MACROS

% Provides a linked \doi{#1} that links doi:#1 to http://dx.doi.org/#1
\usepackage{doi}
% To change the text before the DOI, adjust this command
%\renewcommand\doitext{doi:}

% Provides a linked \url{#1} that doesn't require escape characters
\usepackage{url}

% You can adjust the style \url{} uses here:
% (options are: same, rm, sf, tt; defaults to tt)
\urlstyle{same}

% For \email{ADDRESS}, links ADDRESS to the url mailto:ADDRESS
% (uncomment to typeset the e\-/mail address in typewriter font;
%  otherwise, will be typeset in the \urlstyle above)
%\DeclareUrlCommand\emaillink{\urlstyle{tt}}
\providecommand*\emaillink[1]{\nolinkurl{#1}}
\providecommand*\email[1]{\href{mailto:#1}{\emaillink{#1}}}

\providecommand\BibTeX{{B\kern-.05em{\sc i\kern-.025em b}\kern-.08em \TeX}}
\providecommand\Matlab{\textsc{Matlab}}

% Custom hyphenation rules for words that LaTeX has trouble with
\hyphenation{bio-mim-ic-ry bio-in-spi-ra-tion re-us-a-ble pro-vid-er Media-Wiki}

%%%%%%%%%%%%%%%%%%%%%%%% End Helper Commands %%%%%%%%%%%%%%%%%%%%%%%%%%%

%%%%%%%%%%%%%%%%%%%%%%%%% Begin CV Document %%%%%%%%%%%%%%%%%%%%%%%%%%%%

\begin{document}
\makeheading{~Benjamin~(Min) Wesley~Priest (they/them)}

%%---------------------------------------------------------------------------------------------------------
\section{Contact Information}
%%---------------------------------------------------------------------------------------------------------

% NOTE: Mind where the & separators and \\ breaks are in the following
%       table. Table is one row made up of three parboxes. The left
%       parbox has address info, the middle parbox has a vertical bar,
%       and the right parbox has phone and electronic contact
%       information.
%
% MACROS: \rcollength is the width of the right column of the table
%             (adjust it to your liking; default is 1.85in).
%         \spacewidth is width of area between left and right boxes.
%
\newlength{\rcollength}\setlength{\rcollength}{1.85in}%
\newlength{\spacewidth}\setlength{\spacewidth}{20pt}
%
\begin{tabular}[t]{@{}p{\textwidth-\rcollength-3.5\spacewidth}@{}p{\spacewidth}@{}p{\rcollength}}%
\parbox{\textwidth-\rcollength-\spacewidth}{%
Postdoctoral Researcher\\
\href{http://www.dartmouth.edu/}{Center for Applied Scientific Computing}\\
\href{https://computing.llnl.gov/casc/}{Lawrence Livermore National Laboratory}\\
%14 Engineering Drive\\
%Hanover, NH 03755  USA
}

&
% Uncomment to add a vertical bar in middle of contact information
%{\vrule width 0.5pt}
%\parbox[m][5\baselineskip]{\spacewidth}{} &

% Non-snail-mail contact information
\parbox{1.5\rcollength}{%
\textit{Cell:} +1-937-681-1935 \\
%\textit{Fax:} +1-603-646-3856 \\
\textit{E-mail:} \email{priest2@llnl.gov}\\
%\textit{WWW:} \href{http://www.tedpavlic.com/}{www.tedpavlic.com}
}

\end{tabular}
\vspace{-1em}

%%
%% In modern CV's, it seems like ``Objective'' is frowned upon. Instead,
%% incorporate it into a well-constructed cover letter. The ``More
%% information'' can go at the end of the CV, but it should not distract
%% from the section giving references available to contact.
%%
%
% \section{Objective}
%
% Placement in an academic position (i.e., faculty, postdoctoral, or
% research scientist) that allows for advanced research in distributed
% complex adaptive systems (i.e., modeling, analysis, design, and
% verification) with a particular focus on the control of engineered
% agents (e.g., for communications, control, software, electronics, and
% sustainability) and the analysis of biological phenomena (e.g.,
% self-organization, ecological rationality)
% \begin{innerlist}
% \item More information and auxiliary documents can be found at\\\url{http://www.tedpavlic.com/facjobsearch/}
% \end{innerlist}

%%---------------------------------------------------------------------------------------------------------
\section{Research Interests}
%%---------------------------------------------------------------------------------------------------------
\textbf{Efficient analysis of large, dynamic datasets:}
sketching,
streaming algorithms,
machine learning,
high performance computing,
graph algorithms,
numerical linear algebra,
compressed sensing,
graph theory,
optimization,
network analysis,
%complexity,
%automata,
and theory of deep learning.


\vspace{-0.5em}



%%---------------------------------------------------------------------------------------------------------
\section{Education}
%%---------------------------------------------------------------------------------------------------------
\href{https://engineering.dartmouth.edu/}{\textbf{Thayer School of Engineering}}
at 
\href{https://dartmouth.edu}{\textbf{Dartmouth College}},
Hanover, VT, USA
\vspace{-0.5em}
\begin{outerlist}

\vspace{-0.25em}

\item[] Ph.D.,
             Engineering
             %
\hfill \textbf{09/2015 -- 02/2019}

\vspace{-0.5em}
        \begin{innerlist}
%        \item Thesis Topic: \emph{Design and Analysis of Optimal Task-Processing Agents}
%        \item Thesis Proposal: \emph{Cooperative Task Processing}
%        \item Candidacy: \emph{Research Problems in Distributed Control for Energy Systems}
        \item[-] Advisor:
              \href{http://www.dartmouth.edu/~gvc/}
                   {Professor George~Cybenko}
        \item[-] Thesis: Sublinear Approximations of Vertex Centrality in Evolving Graphs
%        \item[-] Demonstrated novel, sublinear-space sketching algorithms to efficiently estimate local triangle counts and vertex centrality on large, distributed graphs
%        \item[-] Implemented novel HPC algorithms on cutting-edge architectures 
        \end{innerlist}
\end{outerlist}

%\blankline
\vspace{0.5em}


\href{http://www.osu.edu/}{\textbf{The Ohio State University}},
Columbus, OH, USA
\vspace{-0.5em}
\begin{outerlist}

\vspace{-0.25em}
\item[] B.S.,
        \href{http://www.math.osu.edu/}
             {Mathematics}
	\hfill \textbf{09/2007 -- 06/2011}
%        \begin{innerlist}
%        \item \emph{Cum Laude}, With Honors in Mathematics
%        \item Minor in \href{http://www.cse.ohio-state.edu/}
%                            {Philosophy}
%              (logics and formal languages)
%        \end{innerlist}
\vspace{-0.75em}
\item[] B.S.,
        \href{http://www.cse.osu.edu/}
             {Computer and Information Science}
             \hfill \textbf{09/2007 -- 06/2011}

%        \begin{innerlist}
%        \item \emph{Cum Laude}
%        \item Theory specialization (emphasis on computability and complexity)
%        \end{innerlist}

\end{outerlist}

\vspace{-0.75em}


%%---------------------------------------------------------------------------------------------------------
\section{Research Experience}
%%---------------------------------------------------------------------------------------------------------
\href{https://www.llnl.gov/}{\textbf{Lawrence Livermore National Laboratory}},
Livermore, CA, USA

\href{https://computation.llnl.gov/casc/}{\textbf{Center for Applied Scientific Computing}}.
Supervisors:
\href{https://people.llnl.gov/sanders29}{Geoff Sanders},
\href{https://pls.llnl.gov/people/staff-bios/physics/schneider-m}
{Michael Schneider}
and
\href{https://people.llnl.gov/pearce7}{Roger Pearce}

\vspace{-0.5em}

\begin{outerlist}

  \vspace{-0.5em}

  \item[] \textbf{\textit{Computing Scientist}}\hfill \textbf{02/2021 -- present}

  \vspace{-0.5em}

  PI and Co-I of multiple research projects investigating scalable graph
  analytics, machine learning, and statistical modeling on High-Performance
  Computing (HPC) systems.
  Supervised 1 postdoc and 8 graduate students.
  Selected research contributions include novel algorithms and software for
  scalable Gaussian process (GP) estimation~\cite{muyskens2021muygps}, cosmology,
  climate, and space domain modeling~\cite{muyskens2022star}, distributed
  subspace embedding and sketches~\cite{priest2020scaling}, and distributed K
  nearest neighbors.

  \vspace{-0.5em}

  \item[] \textbf{\textit{Postdoctoral Researcher}}%
  \hfill \textbf{04/2019 -- 02/2021}

  \vspace{-0.5em}

  Developed novel sketching algorithms to cluster \cite{priest2020scaling} and
  perform local query approximation \cite{priest2020degreesketch} massive graphs
  on HPC.
  Solved reinforcement learning \cite{goumiri2020reinforcement}, image
  classification \cite{goumiri2020star}, and quantum machine learning
  \cite{otten2020quantum} problems using GPs and neural kernels.

  \vspace{-0.5em}

  \item[] \textbf{\textit{Computation Student Intern}}%
  \hfill \textbf{05/2018 -- 01/2019}

  \vspace{-0.5em}

  Designed novel HPC communication library to accelerate non-traditional
  communications \cite{priest2019you}.
  Used cardinality sketches to estimate local triangle counts in distributed
  graphs \cite{priest2018estimating}.

\end{outerlist}


\halfblankline

\href{https://dartmouth.edu}{\textbf{Dartmouth College}},
Hanover, NH, USA

\href{https://engineering.dartmouth.edu/}{\textbf{Thayer School of Engineering}}.
Advisor:               
\href{http://www.dartmouth.edu/~gvc/}
{Professor George~Cybenko}

\vspace{-0.5em}

\begin{outerlist}

\vspace{-0.5em}

\item[] \textbf{\textit{Research and Teaching Assistant}}%
\hfill \textbf{09/2015 -- 02/2019}
%\vspace{-0.25em}

\vspace{-0.5em}

Invented streaming approximation algorithms for several centrality indices on massive graphs using sketches.
Designed game and graph-theoretic models  for advanced persistent threats in cyber defense.
Taught courses in machine learning and lead a team of TAs.

\end{outerlist}


\halfblankline

\href{http://www.ll.mit.edu/}{\textbf{MIT Lincoln Laboratory}},
Lexington, MA, USA

\href{https://www.ll.mit.edu/mission/cybersec/CADS/CADS.html}
{Cyber Analytics and Decision Systems}.
Supervisor:
\href{https://www.ll.mit.edu/mission/cybersec/cybersec-bios/carter-bio.html}
{Dr. Kevin M.~Carter}

\vspace{-0.5em}

\begin{outerlist}

\vspace{-0.5em}

\item[] {\normalsize \textbf{\textit{Assistant Research Scientist}}}%
\hfill \textbf{08/2011 -- 07/2015}

\vspace{-0.5em}

Modeled computer networks using novel machine learning algorithms.
Developed multi-agent systems for high-fidelity network simulations and cyber defense evaluation.

\end{outerlist}


\halfblankline

\href{http://www.ll.mit.edu/}{\textbf{Air Force Institute of Technology}},
Wright-Patterson Air Force Base, OH, USA

Program Encryption Group.
Supervisor: 
\href{http://www.soc.southalabama.edu/~mcdonald/}
{Professor J.~Todd McDonald}
\vspace{-0.5em}
\begin{outerlist}

\vspace{-0.25em}

\item[] \textbf{\textit{Engineering Technician GS-05}}%
\hfill \textbf{Summer, 2008 \& 2009}

\vspace{-0.5em}

Developed encryption metrics for circuits using abstract interpretation semantic models

\end{outerlist}







%%---------------------------------------------------------------------------------------------------------
\section{Technical \\ Expertise}
%%--------------------------------------------------------------------------------------------------------
\begin{tabular}[t]{@{}p{\textwidth-\rcollength-3.5\spacewidth}@{}p{\spacewidth}@{}p{\rcollength}}%

\parbox{1.5\rcollength}{%
\textbf{Mathematics}
\begin{innerlist}
	\item[] Applied Mathematics
	\item[] Real Analysis
	\item[] Measure Theory
	\item[] Graph Theory
	\item[] Combinatorics
\end{innerlist}
%
\textbf{Computer Science and Engineering}
\begin{innerlist}
	\item[] Distributed \& parallel algorithms
	\item[] Streaming algorithms \& sketching
	\item[] Data structures
\end{innerlist}
\textbf{Data Science and Processing}
%
\begin{innerlist}
    \item[] Probability \& Random Variables
    \item[] Statistics \& Estimation
	\item[] Machine learning \& deep learning
	\item[] Numerical Optimization
    \item[] Stochastic Processes
    \item[] Information Theory
    \item[] Communication Theory
\end{innerlist}
}

&
% Uncomment to add a vertical bar in middle of contact information
%{\vrule width 0.5pt}
%\parbox[m][15\baselineskip]{\spacewidth}{} &

\parbox{\textwidth-\rcollength-\spacewidth}{%
\textbf{Programming and Scripting Languages}
\begin{innerlist}
	\item[] C/C++, Python, Bash, 
	\item[] Julia, Java, R, MATLAB
\end{innerlist}
%%
%\textbf{Scripting Languages}
%\begin{innerlist}
%\end{innerlist}
%%
\textbf{Distributed Computing}
\begin{innerlist}
	\item[] MPI, Hadoop MapReduce, Lustre
\end{innerlist}
%
\textbf{Analytical Software}
\begin{innerlist}
	\item[] Keras, TensorFlow, PyTorch, Mathematica
\end{innerlist}
%
\textbf{Utility Software}
\begin{innerlist}
	\item[] Git, GitHub/Gitlab/Bitbucket
	\item[] \LaTeX{}, \BibTeX
	\item[] Microsoft, LibreOffice, Google Suite 
\end{innerlist}
%
\textbf{Operating Systems}
\begin{innerlist}
    \item[] Apple OS X
    \item[] Linux, RedHat, and other UNIX variants
%    \item[] Microsoft Windows family 
\end{innerlist}
%
\textbf{Interpersonal}
\begin{innerlist}
	\item[] Teamwork and communication
	\item[] Leadership and mentoring
	\item[] Public and technical speaking
\end{innerlist}
}


\end{tabular}
\vspace{0.1in}


%%---------------------------------------------------------------------------------------------------------
\section{Peer-Reviewed Conference Publications}
%%---------------------------------------------------------------------------------------------------------
\begin{itemize}

  \item[\cite{bidese2022light}] \bibentry{bidese2022light}
  \item[\cite{wood2022scalable}] \bibentry{wood2022scalable}
  \item[\cite{goumiri2022light}] \bibentry{goumiri2022light}
  \item[\cite{dunton2022fast}] \bibentry{dunton2022fast}
  \item[\cite{steil2021tripoll}] \bibentry{steil2021tripoll}
  \item[\cite{goumiri2020star}] \bibentry{goumiri2020star}
  \item[\cite{priest2020scaling}] \bibentry{priest2020scaling}
  \item[\cite{goumiri2020reinforcement}] \bibentry{goumiri2020reinforcement}
  \item[\cite{steil2020kronecker}] \bibentry{steil2020kronecker}
  \item[\cite{pearce2019one}] \bibentry{pearce2019one}
  \item[\cite{priest2019you}] \bibentry{priest2019you}
  \item[\cite{steil2019distributed}] \bibentry{steil2019distributed}
  \item[\cite{priest2018estimating}] \bibentry{priest2018estimating}
  \item[\cite{pham2018quantitative}] \bibentry{pham2018quantitative}
  \item[\cite{priest2015agent}] \bibentry{priest2015agent}
  \item[\cite{carter2015characterization}] \bibentry{carter2015characterization}
  \item[\cite{priest2014characterizing}] \bibentry{priest2014characterizing}
  \item[\cite{carter2014latent}] \bibentry{carter2014latent}
  \item[\cite{gold2013modeling}] \bibentry{gold2013modeling}
  \item[\cite{priest2013utility}] \bibentry{priest2013utility}
  \item[\cite{gold2013expectation}] \bibentry{gold2013expectation}

\end{itemize}



%%---------------------------------------------------------------------------------------------------------
\section{Peer-Reviewed Journal Publications}
%%---------------------------------------------------------------------------------------------------------
\begin{itemize}

	\item[\cite{buchanan2022gaussian}] \bibentry{buchanan2022gaussian}
	\item[\cite{muyskens2022star}] \bibentry{muyskens2022star}

\end{itemize}


%%---------------------------------------------------------------------------------------------------------
\section{ArXiv \\Papers}
%%---------------------------------------------------------------------------------------------------------
\begin{itemize}

	\item[\cite{muyskens2021muygps}] \bibentry{muyskens2021muygps}
	\item[\cite{otten2020quantum}] \bibentry{otten2020quantum}
	\item[\cite{priest2020degreesketch}] \bibentry{priest2020degreesketch}

\end{itemize}



%%---------------------------------------------------------------------------------------------------------
\section{Tech \\Reports}
%%---------------------------------------------------------------------------------------------------------
\begin{itemize}

	\item[\cite{filippov2020genetic}] \bibentry{filippov2020genetic}

\end{itemize}



%%---------------------------------------------------------------------------------------------------------
\section{Working \\Papers}
%%---------------------------------------------------------------------------------------------------------
\begin{itemize}

  \item[\cite{boahen2024efficient}] \bibentry{boahen2024efficient}
  \item[\cite{veldt2024metric}] \bibentry{veldt2024metric}
  \item[\cite{priest2024extreme}] \bibentry{priest2024extreme}
  \item[\cite{sallaberry2024scalable}] \bibentry{sallaberry2024scalable}
  \item[\cite{iwabuchi2024communication}] \bibentry{iwabuchi2024communication}
  \item[\cite{priest2023fast}] \bibentry{priest2023fast}
  \item[\cite{muyskens2022analysis}] \bibentry{muyskens2022analysis}

\end{itemize}




%%---------------------------------------------------------------------------------------------------------
\section{Other Conference Publications}
%%---------------------------------------------------------------------------------------------------------
\begin{itemize}

	\item[\cite{priest2017approximating}] \bibentry{priest2017approximating}
	\item[\cite{priest2016efficient}] \bibentry{priest2016efficient}

\end{itemize}



%%---------------------------------------------------------------------------------------------------------
\section{Book Chapters}
%%---------------------------------------------------------------------------------------------------------
\item Geoffrey Sanders, Roger Pearce, \textbf{Bejamin W. Priest}, and Trevor Steil.
	Massive-Scale Distributed Triangle Enumeration and Applications. 
        In:
        David Bader~(Ed.), \emph{Processing Very Large Graphs}, 
%        ch. 11, pp. 232-261. 
%        2019. 
%        Springer.
%        \doi{10.1007/978-1-4614-8806-4_74}
	[In Preparation]
	
\item \textbf{Bejamin W. Priest}, George Cybenko, Satinder Singh, Massimiliano Albanese and Peng Liu.
	Online and Scalable Adaptive Cyber Defense. 
        In:
        Michael Wellman~(Ed.), \emph{Adversarial and Uncertain Reasoning for Adaptive Cyber Defense}, ch.
        11, pp. 232-261. 
        2019. 
        Springer.
%        \doi{10.1007/978-1-4614-8806-4_74}




%%---------------------------------------------------------------------------------------------------------
\section{Conference Talks}
%%---------------------------------------------------------------------------------------------------------
\item \textbf{Benjamin W. Priest}, Alec Dunton, and Geoffrey Sanders.
        Scaling Graph Clustering with Distributed Sketches.
        At: \emph{2020 High Performance Extreme Computing Conference},
        HPEC 2020.
	Waltham, CA, USA (virtual conference),
        21--25 September 2020.

\item \textbf{Benjamin W. Priest} and George Cybenko.
	Approximating centrality in evolving graphs: toward sublinearity.
	At: \emph{2017 SPIE Defense + Security Conference}, 
	SPIE D$\! + \!$S. 
	Anaheim, CA, USA,
        9--13 April 2017.

\item \textbf{Benjamin W. Priest} and George.
	Efficient Inference of hidden Markov models from large observations sequences.
	At: \emph{2016 SPIE Defense + Security Conference}, 
	SPIE D$\! + \!$S. 
	Anaheim, CA, USA,
        17--21 April 2016.

\item \textbf{Benjamin W. Priest}, Era Vuksani and Neal Wagner.
	Agent-based simulation in support of moving target cyber defense technology development and evaluation.
	At: \emph{18th Symposium on Communications \& Networking, 2015 ACM Spring Simulation Multi-Conference}, 
	CNS/SpringSim. 
	Alexandria, VA, USA,
        12--15 April 2015.

\item \textbf{Benjamin W. Priest} and Kevin M. Carter.
	Characterizing latent user interests on enterprise networks.
	At: \emph{2014 International Florida Artificial Intelligence Research Society Conference},
	FLAIRS.
	Pensacola Beach, FL, USA,
	21--23 May 2014.



%%---------------------------------------------------------------------------------------------------------
\section{Invited Talks}
%%---------------------------------------------------------------------------------------------------------
\begin{innerlist}

\item[] \myname.
	High-fidelity enterprise network emulation using the GOSMR architecture.
	In: \emph{2014 MIT Lincoln Laboratory Cyber and Net-Centric Workshop}, CNW.
	June, 2014.
	
\end{innerlist}



%%---------------------------------------------------------------------------------------------------------
\section{Poster Presentations}
%%---------------------------------------------------------------------------------------------------------
\begin{innerlist}

\item[] \myname.
	Estimating edge-local triangle count heavy hitters in edge-linear time and almost-vertex-linear space.
	At: \emph{GraphChallenge Workshop at the IEEE High Performance Extreme Computing Conference}, 
	HPEC. 
	25--27 September 2018.

\item[] \myname.
	Efficient Sublinear Estimation of Local Triangle Count Heavy Hitters.
	At: \emph{2018 Summer Student Poster Symposium at Lawrence Livermore National Laboratory}. 
	9 August 2018.

\item[] \myname.
	Characterization of latent social networks discovered through computer network logs.
	At: \emph{Networks in the Social and Information Sciences workshop of the 29th Annual Conference on Neural Information Processing Systems},
	NIPS.
	Montreal, Canada, 
	12 December 2015.

\item[] \myname.
	Utility discounting explains informational website traffic patterns before a hurricane.
	At: \emph{22nd International World Wide Web Conference}, 
	WWW. 
	2013.
	Rio de Janeiro, Brazil,
	13--17 May 2013.

\end{innerlist}



%%---------------------------------------------------------------------------------------------------------
\section{Grants}
%%---------------------------------------------------------------------------------------------------------

\textbf{In Preparation}

\begin{innerlist}

  \item[] N/A

\end{innerlist}


\blankline

\textbf{Awarded}

\begin{innerlist}

\item Co-I,
	``MuyGPs: Non-Stationary Gaussian Processes at HPC Scales'',
	LLNL LDRD ER,
	\$850,000/year.
	October 1,~2021 to September 30,~2024.

\item Co-I,
	``Scalable Uncertainty Quantification Using Gaussian Processes Surrogate Models'',
	LLNL LDRD 21-FS-037,
	\$100,000.
	January 1,~2021 to September 30,~2021.

\item Co-I,
        ``Interactive Exploratory Graph-Enabled Data Analytics at HPC Scales'',
        LLNL LDRD 21-ER-020,
        \$500,000/year.
        October 1,~2020 to September 30,~2022.

\item Co-PI,
	``Scalable Approximate Graph Clustering'',
	LLNL LDRD 20-FS-037,
	\$150,000.
	February 1,~2020 to September 30,~2020.

\end{innerlist}


\blankline

\textbf{Declined}

\begin{innerlist}

  \item[] \textbf{PI},
  ``Efficient Data Reduction via Active Learning Augmented Coresets'',,
  \begin{innerlist}
    \item[-] DOE ASCR DE-FOA-0003266
    \item[-] \$1,000,000/yr.
    \item[-] FY 25-27.
  \end{innerlist}

  \item[] \textbf{Co-I},
  ``Scalable Physics-Constrained Emulation of Stochastic Moments of Quantum Processing Units'',
  \begin{innerlist}
    \item[-] DOE ASCR DE-FOA-0003300
    \item[-] \$500,000/yr.
    \item[-] FY 25-26.
  \end{innerlist}

  \item[] \textbf{Co-I},
  ``Science with LSST Year 1 Data'',
  \begin{innerlist}
    \item[-] LLNL LDRD ER
    \item[-] \$1,000,000/yr.
    \item[-] FY 25-27.
  \end{innerlist}

  \item[] \textbf{Co-I},
  ``Decentralized Strategic Decision Making via Linearly Solvable Markov Games'',
  \begin{innerlist}
    \item[-] LLNL LDRD ER
    \item[-] \$600,000/yr.
    \item[-] FY 25-27.
  \end{innerlist}

  \item[] \textbf{Co-I},
  ``Uncertainty-Aware Realtime Algorithmic Calibration in Distributed Computing Environments'',
  \begin{innerlist}
    \item[-] DOE ASCR DE-FOA-0003266
    \item[-] \$1,200,000/yr.
    \item[-] FY 24-27.
  \end{innerlist}

  \item[] \textbf{PI},
  ``Scalable and Highly Concurrent Sampling Algorithms for Scale-Free Graphs'',
  \begin{innerlist}
    \item[-] DOE ASCR DE-FOA-0002722
    \item[-] \$800,000/yr.
    \item[-] FY 23-25.
  \end{innerlist}

  \item[] \textbf{Co-I},
  ``Probabilistic AI Pipeline Modules for Rubin LSST Dark Energy'',
  \begin{innerlist}
    \item[-] DOE ASCR DE-FOA-0002705
    \item[-] \$1,000,000/yr.
    \item[-] FY 23-25.
  \end{innerlist}

  \item[] \textbf{Co-PI},
  ``Scalable Single-Pass Compressive Autoencoders'',
  \begin{innerlist}
    \item[-] LLNL LDRD Feasibility Study
    \item[-] \$150,000/yr.
    \item[-] FY 23.
  \end{innerlist}

  \item[] \textbf{Co-I},
  ``EpochGrafts: Relational Data Fusion via Dynamic Graph Analysis'',
  \begin{innerlist}
    \item[-] LLNL LDRD ER
    \item[-] \$500,000/yr.
    \item[-] FY 22-24.
  \end{innerlist}

  \item[] \textbf{PI},
  ``Scalable Non-stationary Approximate Gaussian Processes'',
  \begin{innerlist}
    \item[-] DOE ASCR DE-FOA-0002493
    \item[-] \$800,000/yr.
    \item[-] FY 22-24.
  \end{innerlist}

  \item[] \textbf{PI},
  ``Distributed Memory Sketching Algorithms at HPC Scales'',
  \begin{innerlist}
    \item[-] DOE ASCR DE-FOA-0002497
    \item[-] \$400,000/yr.
    \item[-] FY 22-23.
  \end{innerlist}

  \item[] \textbf{PI},
  ``\texttt{croquis}: Distributed Subspace Embeddings for High Performance Computing'',
  \begin{innerlist}
    \item[-] LLNL ISCP Tech Base
    \item[-] \$100,000/yr.
    \item[-] FY 21-22.
  \end{innerlist}

\end{innerlist}



%%---------------------------------------------------------------------------------------------------------
\section{Mentorship}
%%---------------------------------------------------------------------------------------------------------

\textbf{Postdocs}
%
\begin{innerlist}

  \item[] \textbf{Grace Li},
  Postdoc in Applied Mathematics.
  \begin{innerlist}
    \item[-] Lawrence Livermore National Laboratory
    \item[-] Distributed memory graph algorithms, clustering, and analysis.
    \item[-] Exascale energy grid expansion modeling under topological uncertainty.
    \item[-] 2024-current.
  \end{innerlist}

  \item[] \textbf{Alec Dunton},
  Postdoc in Applied Mathematics.
  \begin{innerlist}
    \item[-] Lawrence Livermore National Laboratory
    \item[-] Fast and scalable Gaussian process approximation in distributed memory.
    \item[-] 2021-2023.
  \end{innerlist}

\end{innerlist}


\blankline

\textbf{Students}
%
\begin{innerlist}

  \item[] \textbf{Edem Boahen},
  PhD student in Mathematics,
  Michigan State University.
  Clustering guarantees for power iteration algorithms on degree-corrected
  stochastic block models.
  Primary Advisor: Mark Iwen.
  Summer 2024-current.

  \item[] \textbf{Sunanda Thirunabukkarasu (Mila Arasu)},
  BS student in Astrophysics,
  Arizona State University.
  Simulating cosmic lensing shear with realistic B- and E-mode shifts.
  Summer 2024.

  \item[] \textbf{Lance Fletcher},
  PhD student in Computer Science,
  Texas A \& M University.
  Massively parallel random walk sampling and embeddings.
  Primary Advisor: Roger Pearce.
  Autumn 2023-current.

  \item[] \textbf{Juliette Mukangango},
  PhD student in Statistics,
  Colorado School of Mines.
  Novel loss and objective functions for outlier robustness in training sparse
  MuyGPs models.
  Primary Advisor: Douglas Nychka.
  Summer 2023.

  \item[] \textbf{Akil Andrews},
  PhD student in Computer Science,
  University of New Mexico.
  Adaptive Bayesian optimization under changing data representations.
  Primary Advisor: Melanie E Moses.
  Summer 2023.

  \item[] \textbf{Ian McGovern},
  PhD student in Statistics,
  University of California, Los Angeles.
  Uncertainty analysis of hybrid deep neural network and Gaussian process
  predictions.
  Primary Advisor: Frederic Schoenberg.
  Summer 2023.

  \item[] \textbf{Abiodun Sumonu},
  PhD student in Mathematics,
  University of Alabama.
  Survey of Biclustering Algorithms.
  Summer 2023.

  \item[] \textbf{Keegan Kresge},
  Post Baccalaureate in Mathematics,
  DOD.
  K-Nearest Neighbors performance of exponentiated subspace embeddings on large
  graphs.
  Summer 2023.

  \item[] \textbf{Marina Dunn},
  Masters student in Data Science,
  University of California, Riverside.
  Visualizing sparse Gaussian process optimization.
  Summer 2022.

  \item[] \textbf{Killian Wood},
  PhD Student in Applied Mathematics,
  University of Colorado, Boulder.
  Multiscale Bayesian optimization of MuyGPs.
  Primary Advisor: Emiliano Dall'Anese.
  Summer 2022.

  \item[] \textbf{Micha{\l}
    Lisicki},
  PhD student in Computer Science,
  University of Guelph.
  Distributional reinforcement learning on gridworld environments.
  Primary Advisor: Graham Taylor.
  Summer 2022.

  \item[] \textbf{Sudharshan Srinivasan},
  PhD student in Computer Science,
  University of Oregon.
  Communication optimization for highly non-uniform distributed graph algorithms.
  Primary Advisor: Boyana Norris.
  Summer 2021.

  \item[] \textbf{Alec Dunton},
  PhD student in Applied Mathematics,
  University of Colorado, Boulder.
  Parameter sensitivity of stochastic block models under subspace embeddings.
  Primary Advisor: Alireza Doostan.
  Summer 2020.

\end{innerlist}


\blankline

\textbf{Teams and Challenges}
%
\begin{innerlist}

  \item[] \textbf{Davy Walker, Ukamaka Nnyaba, and Hewan Shemtaga},
  \begin{innerlist}
    \item[-] PhD Students in Computer Science, Auburn University.
    \item[-] Auburn Data Science Capstone Project.
    \item[-] ECG Time Series Classification Using Computationally Efficient Gaussian Processes.
    \item[-] Autumn 2023.
  \end{innerlist}

  \item[] \textbf{Rafael Bidese, Chinedu Eleh, and Yunli Zhang},
  \begin{innerlist}
    \item[-] PhD Students in Computer Science, Auburn University.
    \item[-] Auburn Data Science Capstone Project.
    \item[-] Stellar Blend Image Classification Using Computationally Efficient Gaussian Processes.
    \item[-] Autumn 2022.
  \end{innerlist}

  \item[] \textbf{Jocelyn Ornelas, Alan Triano, Cristian Espinosa, Denylson Fuentes, and Rahul Ravi},
  \begin{innerlist}
    \item[-] A PhD Students (Jocelyn) and 4 undergraduates in Data Science programs.
    \item[-] University of California, Merced.
    \item[-] LLNL Data Science Challenge.
    \item[-] Asteroid detection and orbit extraction from astronomy corpora.
    \item[-] Sprint 2021.
  \end{innerlist}

\end{innerlist}


%%---------------------------------------------------------------------------------------------------------
\section{Professional Service}
%%---------------------------------------------------------------------------------------------------------

\textbf{Conference Service}
%
\begin{innerlist}
  \item[-] Program Committee: SIAM International Conference on Data Mining (SDM24).
  Houston, Texas, USA.
  April 18-20, 2024.

  \item[-] Program Committee: 29th ACM SIGKDD Conference on Knowledge Discovery and Data Mining (KDD23).
  Long Beach, California, USA.
  August 6-10, 2024.

  \item[-] Program Committee: 28th International AAAI Florida Artificial Intelligence Research Symposium Conference,
  FLAIRS-28.
  Hollywood, Florida, USA.
  May 18-20, 2015.
\end{innerlist}


\textbf{Journal Service}
%
\begin{innerlist}
  \item[-] Reviewer: American Astronomical Society: The Astrophysical Journal.
  2022.
\end{innerlist}


\textbf{Committee Service}

\begin{innerlist}
  \item[-]: Committee Member.
  LLNL/Computing Inclusion, Diversity, Equity, and Accountability Committee.
  Subcommittees: ``Improving Student Pipelines'' and ``Thrive Conversations''.  
  2023-present.

  \item[-] Leader/Organizer.
  LLNL/Computing Summer SLAM! presentation session for summer students.
  Summer, 2024.

  \item[-] Leader/Organizer.
  LLNL/Computing Summer SLAM! presentation session for summer students.
  Summer, 2023.

  \item[-] Committee Member.
  LLNL/Computing Stategic Initiative for DEI in Recruiting, Outreach, and Hiring.
  2021.

\end{innerlist}



\textbf{Mentorship Service}

\begin{innerlist}
  \item[-]: Instructor: ``DevOps for Data Scientists''.
  LLNL Data Science Summer Institute.
  Summer, 2024.

  \item[-]: Team Mentor.
  Auburn University Data Science Capstone project.
  ``Enhancing Electrocardiography Data Classification Confidence: A Robust Gaussian Process Approach''
  Autumn, 2023.

  \item[-]: Instructor: ``DevOps for Data Scientists''.
  LLNL Data Science Summer Institute.
  Summer, 2023.

  \item[-]: Team Mentor.
  Auburn University Data Science Capstone project.
  ``Stellar Blend Image Classification Using Computationally Efficient Gaussian Processes'',
  Autumn, 2022.

  \item[-]: Instructor: ``The Streaming Data Model and Selected Algorithms''.
  LLNL Data Science Summer Institute.
  Summer, 2022.

  \item[-]: Team Coach.
  LLNL/UC Merced Data Science Challenge.
  Summer, 2022.

\end{innerlist}






%%---------------------------------------------------------------------------------------------------------
\section{Teaching Experience}
%%---------------------------------------------------------------------------------------------------------
\href{https://engineering.dartmouth.edu/}{\textbf{Thayer School of Engineering}}
at 
\href{https://dartmouth.edu}{\textbf{Dartmouth College}},
Hanover, VT, USA

\begin{innerlist}
\item[] \textit{Teaching Assistant}%
    \begin{innerlist}
        \item[] Instructor for ENGS/QBS 108: Applied Machine Learning
            \hfill \textbf{Autumn 2017}
        \begin{innerlist}
            \item[-] Collaborated with instructors to develop course curriculum aimed at graduate engineering and computer science students and taught $\sim 25\%$ of course lecture content.
            \item[-] Led team of 4 teaching assistants
            \item[-] Provided group and one-on-one assistance to students covering lecture topics
            \item[-] Planned, wrote, and graded all student assignments
        \end{innerlist}

%        \halfblankline

        \item[] Instructor for ENGS 177: Decision Making Under Risk and Uncertainty
            \hfill \textbf{Winter 2017}
        \begin{innerlist}
            \item[-] Planned and taught a weekly recitation covering practical machine learning topics
            \item[-] Provided ground and one-on-one assistance to students covering lecture topics
            \item[-] Wrote student assignments with the assistance of the instructor and provided grading
        \end{innerlist}

    \end{innerlist}

\end{innerlist}


\href{http://www.osu.edu/}{\textbf{The Ohio State University}},
Columbus, OH, USA
\begin{innerlist}

\item[] \textit{Teaching Assistant}%
    \begin{innerlist}
        \item[] Instructor for CSE~625: Automata and Formal Languages
            \hfill \textbf{Summer \& Autumn 2010}
        \begin{innerlist}
            \item[-] Planned and taught a weekly recitation covering details and proofs of lecture topics
            \item[-] Graded student assignments
        \end{innerlist}

%        \halfblankline

        \item[] Grader for CSE~560: System Software Design and Devlopment
            \hfill \textbf{Summer 2010}
        \begin{innerlist}
            \item[-] Graded student assignments and held office hours
        \end{innerlist}

    \end{innerlist}

\end{innerlist}


%%---------------------------------------------------------------------------------------------------------
\section{Awards}
%%---------------------------------------------------------------------------------------------------------
\begin{tabular}[t]{@{}p{\textwidth-\rcollength-3.5\spacewidth}@{}p{\spacewidth}@{}p{\rcollength}}%

% Address box
\parbox{\textwidth-\rcollength-\spacewidth}{%
\href{https://graphchallenge.mit.edu/champions}{HPEC Graph Challenge}
\begin{innerlist}
\item[] Graph Challenge Champion, 2020.
\item[] Graph Challenge Champion, 2019.
\end{innerlist}

\halfblankline

\href{http://www.secrypt.icete.org/PreviousAwards.aspx}{SECRYPT}
\begin{innerlist}
\item[] Best Paper Award, 2018.
\end{innerlist}

\halfblankline

\href{http://www.ll.mit.edu}{MIT Lincoln Laboratory}
\begin{innerlist}
\item[] Lincoln Scholar Fellowship, 2015
\end{innerlist}
%14 Engineering Drive\\
%Hanover, NH 03755  USA
}

&
% Uncomment to add a vertical bar in middle of contact information
%{\vrule width 0.5pt}
%\parbox[m][5\baselineskip]{\spacewidth}{} &

% Non-snail-mail contact information
\parbox{1.5\rcollength}{%
\href{http://www.osu.edu}{The Ohio State University}
\begin{innerlist}
\item[] Phi Beta Kappa Inductee, 2010
\item[] Bingham Award in Philosophy, 2010
\item[] Kenneth Cummings Scholarship, 2008--2011
\item[] Distinguished Merit Scholarship, 2007--2011
\item[] Ohio Academic Scholarhship, 2007-2011
\end{innerlist}
}

\end{tabular}






%%---------------------------------------------------------------------------------------------------------
 \section{Security Clearance}
%%---------------------------------------------------------------------------------------------------------

DOE Q-clearance (Spring 2024).

%%---------------------------------------------------------------------------------------------------------
 \section{Citizenship}
%%---------------------------------------------------------------------------------------------------------

 USA


\bibliographystyle{unsrt}
\nobibliography{../bibs/pubs.bib}

\end{document}

%%%%%%%%%%%%%%%%%%%%%%%%%% End CV Document %%%%%%%%%%%%%%%%%%%%%%%%%%%%%

%----------------------------------------------------------------------%
% The following is copyright and licensing information for
% redistribution of this LaTeX source code; it also includes a liability
% statement. If this source code is not being redistributed to others,
% it may be omitted. It has no effect on the function of the above code.
%----------------------------------------------------------------------%
% Copyright (c) 2007, 2008, 2009, 2010, 2011 by Theodore P. Pavlic
%
% Unless otherwise expressly stated, this work is licensed under the
% Creative Commons Attribution-Noncommercial 3.0 United States License. To
% view a copy of this license, visit
% http://creativecommons.org/licenses/by-nc/3.0/us/ or send a letter to
% Creative Commons, 171 Second Street, Suite 300, San Francisco,
% California, 94105, USA.
%
% THE SOFTWARE IS PROVIDED "AS IS", WITHOUT WARRANTY OF ANY KIND, EXPRESS
% OR IMPLIED, INCLUDING BUT NOT LIMITED TO THE WARRANTIES OF
% MERCHANTABILITY, FITNESS FOR A PARTICULAR PURPOSE AND NONINFRINGEMENT.
% IN NO EVENT SHALL THE AUTHORS OR COPYRIGHT HOLDERS BE LIABLE FOR ANY
% CLAIM, DAMAGES OR OTHER LIABILITY, WHETHER IN AN ACTION OF CONTRACT,
% TORT OR OTHERWISE, ARISING FROM, OUT OF OR IN CONNECTION WITH THE
% SOFTWARE OR THE USE OR OTHER DEALINGS IN THE SOFTWARE.
%----------------------------------------------------------------------%
